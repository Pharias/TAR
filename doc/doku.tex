\documentclass[12pt,a4paper]{article}
\usepackage {ngerman}
\usepackage{helvet}
\renewcommand{\familydefault}{\sfdefault}
\usepackage{amsmath}
\usepackage{amsfonts}
\usepackage{amssymb}
\usepackage{graphicx}
\usepackage[left=3cm,right=2.5cm,top=2.5cm,bottom=2.5cm]{geometry}
\usepackage[headsepline,footsepline]{scrlayer-scrpage}
\usepackage{microtype}
\usepackage{pdfpages}
\usepackage[hyphens]{url}
\usepackage[colorlinks=true,urlcolor=blue,linkcolor=black]{hyperref}
\hyphenpenalty=9999
\exhyphenpenalty=9999
\setheadsepline{.5pt}
\setfootsepline{.5pt}
\setlength{\headheight}{20pt} 
\pagestyle{scrheadings}
\clearscrheadfoot
\clearscrheadings
\clearscrplain
\author{Felix Kuschel
	\and Manuel Starz}
\title{Entwicklung einer Smart Home Zentrale auf Basis eines RaspberryPI}
% Kopf- & Fußzeilen... werden nicht angezeigt, wenn \chapter{} verwendet wird. Stattdessen \section{} verwenden...
% Kopfzeile
\ihead{\includegraphics[width=0.2\textwidth]{its_logo.pdf}}
\chead{}
\ohead{\textsc{Kuschel \& Starz}}
% Fußzeile
\ifoot{\textsc{FTI2 2020/2021}}
\cfoot{}
\ofoot{\pagemark}
\date{-}
\begin{document}
	% Titelseite. Basierend auf https://de.wikibooks.org/wiki/LaTeX/_Eine_Titelseite_erstellen
	\begin{titlepage}
		\centering
		\includegraphics[width=0.5\textwidth]{its_logo.pdf}\par\vspace{1cm}
		% {\scshape\LARGE it.schule stuttgart \par}
		\vspace{1cm}
		{\scshape\Large Technikerarbeit\par}
		\vspace{1.5cm}
		{\huge\bfseries\scshape Entwicklung einer Smart Home Zentrale auf Basis eines RaspberryPI\par}
		\vspace{2cm}
		{\Large Felix \textsc{Kuschel} \& Manuel \textsc{Starz}\par}
		\vfill
		Betreut durch\par
		~Matthias \textsc{Kohler}

		\vfill

		% Bottom of the page
		{\large \today\par}
	\end{titlepage}
	% Inhaltsverzeichnis
	\tableofcontents
	\newpage
	\section*{Erklärung}
	Wir versichern, dass die vorliegende Abschlussarbeit von uns selbstständig angefertigt und nur die angegebenen Hilfsmittel benutzt wurden. An Stellen, die dem Wortlaut oder dem Sinne nach anderen Werken entnommen sind, haben wir dies durch die Angabe der Quellen kenntlich gemacht.
	\vspace{2cm}	
	\\
	\noindent\rule{7cm}{.4pt}\hfill\rule{7cm}{.4pt}\par
	\noindent Datum, Ort \hfill Felix Kuschel
	\vspace{2cm}	
	\\
	\noindent\rule{7cm}{.4pt}\hfill\rule{7cm}{.4pt}\par
	\noindent Datum, Ort \hfill Manuel Starz
	\newpage
	\section{Vorwort}
 	\subsection{Einleitung}
 	Bei dieser Ausarbeitung handelt es sich um die Abschlussarbeit zur Weiterbildung zum staatlich geprüften Techniker in der Fachrichtung Informationstechnik. Diese Arbeit basiert auf dem in den zwei Jahren erlernten Stoffs sowie selbst erarbeiteten Kenntnissen und dient zur Feststellung des Erreichen des Fortbildungsziels.
 	\subsection{Projektrahmen}
 	Das Projekt zur Erstellung einer Smart Home Zentrale wurde von uns, Felix Kuschel und Manuel Starz, durchgeführt. Der Projektzeitraum war vom 1. September 2020 bis zum 30.03.2021 angesetzt. Es stand zur Umsetzung des Projekts ein unterrichtsfreier Tag pro Woche zur Verfügung.\par
 	Des weiteren wurden die in dem Zeitraum zur Verfügung stehenden Ferien zur Umsetzung des Projekts genutzt. Die Projektbetreuung erfolgten seitens der Schule durch Herr Matthias Kohler. Da das Projekt nicht in Zusammenarbeit mit einem Unternehmen durchgeführt wurde, gibt es keine weiteren Betreuer. Die Materialkosten für die im Projekt genutzte Hardware wurde von den uns selbst getragen.
 	\subsection{Aufgabenstellung}
 	Das Projekt ist aus dem Zusammenschluss der Abschlussarbeitsideen von uns, Felix Kuschel und Manuel Starz, entstanden und wurde mit Rücksprache mit dem betreuenden Lehrer entwickelt.\par
 	 Durch die große Verfügbarkeit von Smart Home Geräten und den zahlreichen Standards der Anbieter entschlossen wir uns, eine einfache und leicht zu replizierende Lösung zu entwickeln, die Smart Home Geräte mehrerer Hersteller miteinander verknüpft und so die Notwendigkeit mehrerer verschiedener sogenannter Hubs zu eliminieren. \par
 	Des weiteren soll das Gerät noch über einen Touchscreen steuerbar sein und die Werte der verbundenen Smart Home Geräte anzeigen. Dies umfasst unter anderem den Status von Leuchtmitteln, die Werte von Thermostaten sowie den Zustand von Tür- und Fensterkontakten. Die genaue Aufgabenstellung kann dem abgegebenen Lastenheft im Anhang entnommen werden.
 	\subsection{Zusatzinformationen}
 	Die Rohdaten des Projekts wurden der Einfachheit in einem GIT-Projekt zusammengefasst. Dadurch konnten die Durchführenden unabhängig voneinander an dem Projekt und der Dokumentation arbeiten.\par
 	Der Link zu dem Projekt lautet:\\
 	\textbf{https://github.com/Pharias/TAR}\par
 	Ursprünglich wurde für die Verwendung des schulinternen GIT verwendet. Dies stand zum Zeitpunkt der Erstellung der Dokumentation allerdings nicht zur Verfügung, weshalb eine Alternative genutzt wurde.
 	\newpage
	% Hier den Rest eintragen... 	
 	\section{Quellen}
 	% Quellen auflisten
 	\subsection{Verwendete Software}
 	\begin{tabular}{|l|l|l|}
 	\hline 
 	\textbf{Software} & \textbf{Verwendung} & \textbf{Version} \\ 
 	\hline 
 	Raspberry Pi OS & Betriebssystem und Oberfläche für Hardware & 5.4 \\ 
 	\hline 
 	Home Assistant & Betriebssystem und Oberfläche für Smart Home & 5.12 \\ 
 	\hline 
 	KiCad EDA & Erstellung von Schaltplan und Gerber-Datei des Hats & 5.1.8 \\ 
 	\hline 
 	TexMaker & Erstellung der Dokumentation & 5.0.4 \\ 
 	\hline 
 	• & • & • \\ 
 	\hline 
 	\end{tabular} 
 	\subsection{Verwendete Hardware}
 	\begin{tabular}{|l|l|l|}
 	\hline
 	\textbf{Hardware} & \textbf{Verwendung} & \textbf{Version} \\
 	\hline
 	Raspberry Pi & Hauptplatine für die  & Version 4B (8GB)\\
 	 & Smart Home Zentrale &\\
 	\hline
 	CC2531 Zigbee USB & USB-Stick mit ZigBee-Chip & Rev 2.4\\
 	Stick mit Firmware& & \\
 	\hline
 	Sunfounder 10.1  & Bildschirm und Input für & Unbekannt\\
 	Touch Screen & die Smart Home Zentrale & \\
 	 \hline
 	\end{tabular}
 	\section{Anhang}
 	% Lastenheft, andere Dokumente etc...
 	\subsection{Lastenheft}
 	\includegraphics[width=0.98\textwidth, page=1]{lastenheft.pdf}
 	\newpage
 	\includegraphics[width=0.98\textwidth, page=2]{lastenheft.pdf}
 	\newpage
 	\subsection{Hardware-Dokumentationen}
 	Durch den Umfang der einzelnen Dokumentationen hier nur eine Auflistung der Dokumentationen mit dem Link zu den PDF im GIT-Projekt bzw. den Herstellerseiten.
 	\begin{itemize}
 		\item Raspberry Pi:\\ {\url{https://www.raspberrypi.org/documentation/hardware/raspberrypi/bcm2711/rpi_DATA_2711_1p0.pdf}}
 		\item MiFare MFRC522:\\ {\url{https://www.nxp.com/docs/en/data-sheet/MFRC522.pdf}}
 		\item CC2531 ZigBee SoC:\\ {\url{https://www.ti.com/lit/ds/symlink/cc2531.pdf}}
 		\item ATmega128RFA1-ZU:\\ {\url{https://ww1.microchip.com/downloads/en/DeviceDoc/Atmel-8266-MCU_Wireless-ATmega128RFA1_Datasheet.pdf}}
 	\end{itemize}
\end{document}
