\subsection{Kostenaufstellung}\label{ku_kosten}
Nachfolgen haben wir die Kosten des Projekts aufgelistet.\par
\subsubsection{Beschaffungskosten}
\begin{center}
\begin{tabularx}{\textwidth}{|p{5.6cm}|p{1.2cm}|p{3.5cm}|p{3.5cm}|}
 	\hline
 	\textbf{Produkt} & \textbf{Menge} & \textbf{Kosten/Stk}  & \textbf{Kosten/Gesamt}\\
	\hline
	Raspberry Pi 4 Modell B 8GB & 1 & 87,22\euro{} & 87,22\euro{} \\
	\hline
	SanDisk Extreme microSD 128 GB & 1 & 19,99\euro{} & 19,99\euro{} \\
	\hline
	SUNFOUNDER RPi 10.1'' Touch Display & 1 & 129,99\euro{} & 129,99\euro{} \\
	\hline
	Noctua NF-A4x10 5v & 1 & 12,90\euro{} & 12,90\euro{} \\
	\hline
	ITSTUFF CC2531 Zigbee USB-Stick & 1 & 14,90\euro{} & 14,90\euro{} \\
	\hline
	ITSTUFF Alu-Kühlkörper Set RPi 4 & 1 & 4,91\euro{} & 4,91\euro{} \\
	\hline
	Eightwood SMA Verlängerung & 1 & 7,99\euro{} & 7,99\euro{} \\ 
	\hline
	VCE 5,5mm Stecker \& Buchse & 1 & 8,29\euro{} & 8.29\euro{} \\
	\hline
	dasFilament PTGE schwarz 1,75mm & 1 & 24,95\euro{}\footnote{Kosten der Spule (800g).} & 11,61\euro{}\footnote{Kosten verwendetes Material für Gehäusedruck} \\
	\hline
	USB-Verlängerung & 1 & 9,99\euro{} & 9,99\euro{}\\
	\hline
	% hier noch die Kosten für das Zeug, das du gekauft hast, Felix. der Stick steht ja schon drin, aber weiß nicht ob der Preis passt...
	%\hline
	\textbf{Gesamtkosten} &  &  & \textbf{XXX\euro{}} \\ % Kosten zusammenrechnen ;)
	\hline
\end{tabularx}
\end{center}

\subsubsection{Kostenberechung für die 3D-Druckteile}
\noindent Die Kosten lassen mit folgender Formel darstellen:\\
\[
K_{gesamt} =  (K_{Material/g} \cdot M_{Material}) +  (T_{Druck} \cdot  K_{kWh} \cdot V_{\emptyset/h})
\]
Die Kosten der 3D-Druckteile lassen sich in zwei Einzelpositionen, Material und Energie.
Die Materialkosten lassen sich leicht berechnen, da wir die Kosten der Spule Filament mit 24.95\euro{} und einem Gewicht von 800g Filament auf der Spule.
Dies bedeutet einen Preis von 0,02(74...)\euro{} ($K_{Material/g}$) pro Gramm verwendetem Filament.
Der Druck der Gehäuseteile verbraucht 423g Material ($M_{Material}$), dass heißt, dass die Materialkosten sich auf 11,61\euro{} belaufen.\par
\noindent Die Energiekosten zu berechnen ist etwas komplexer.
Der 3D-Drucker verbraucht circa 120W pro Stunde($V_{\emptyset/h}$)\footnote{Robert (2019): Antwort auf Creality Ender 3 printer power consumption? - 3dprinting Stack Exchange}, was bei einem Strompreis von 22,3\euro{}ct / kWh ($K_{kWh}$) bedeuten würde, dass wir pro Stunde etwa 2,676\euro{}ct Stromkosten haben. 
Bei einer Gesamtdruckzeit von 36 Stunden und 36 Minuten($T_{Druck}$) ergibt sich also Stromkosten von 97,9416\euro{}ct.\par
\noindent Zählt man nun die Kosten für Material und Energie zusammen, ergibt sich ein Fertigungspreis von 12,59\euro{} ($K_{gesamt}$). 
Dabei ist die Veredelung (Schleifen \& Lackieren) noch nicht mit einkalkuliert.\par
