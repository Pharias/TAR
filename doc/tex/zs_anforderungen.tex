\subsection{Anforderungen und gewünschte Features}
Die Anforderungen an das Projekt lauteten demnach wie folgt:
\begin{itemize}
	\item Anbindung von ZigBee-fähigen Endgeräten
	\item Steuerung der angebundenen Endgeräte
	\item Übermittlung der Zustände der angebundenen Geräte an z.B. ein Smartphone
\end{itemize}
Diese Anforderungen lassen sich mit einem Raspberry Pi und einem Zigbee-USB-Stick realisieren. Darüber hinaus waren von unserer Seite noch folgende Features gewünscht:
\begin{itemize}
	\item Ein- und Ausgabe über einen Touch-Bildschirm
	\item Einbindung eines Sprachassistenten zur Steuerung der eingebundenen Endgeräte
\end{itemize}
Nach Fortschritt des Projekts kam bei einer Rücksprache mit unserem Projektbetreuer die Idee auf, einen Raspberry Pi Hat speziell für das Projekt zu entwickeln. Dieser sollte die Hardware des Projekts falls möglich auf einer Platine vereinen, die dann auf den Raspberry Pi aufgesteckt werden konnte.\par
Diese Erweiterungsplatine sollte folgende Eigenschaften besitzen:
\begin{itemize}
	\item ZigBee-Controller und Antenne
	\item NFC-Controller und Antenne
	\item RGB-LED zur Statusanzeige
	\item Anschluss für Lüfter
	\item Sensoren für:
	\begin{itemize}
		 \item Luftfeuchtigkeit
		 \item Temperatur
		 \item Luftdruck
	\end{itemize}
	\item Pins zum Anschluss an Versuchsaufbau für Laborgebrauch
\end{itemize}
Die Erweiterungsplatine wurde aber wie zuvor aufgrund der aktuellen Pandemie-Situation und den damit verbundenen Beschaffungsschwierigkeiten verworfen.
Darauf wurde dann klar, dass eine Erstellung eines Installationsskripts für den Raspberry Pi eine sinnvolle Ergänzung der Projektarbeit wäre. Zusätzlich haben wir ein Gehäuse für die Hardware geplant, um das Endprodukt so wertiger gestalten zu können.