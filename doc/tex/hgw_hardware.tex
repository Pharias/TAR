\subsection{Hardware}\label{hgw_hardware}
Bei der Entwicklung der Hardware haben wir für unser Projekt intern einige Rahmenbedinungen festgelegt:
\begin{itemize}
	\item Das Projekt sollte nach Möglichkeit von durchschnittlichen Bastlern durchgeführt werden
	\item Zur Fertigstellung des Projekts sollte keine weiteren Werkzeuge als die bei uns persönlich vorhandenen Werkzeuge vorhanden sein
	\item Die Grundplattform sollte der 2020 herausgebrachte Raspberry Pi 4 8GB sein\footnote{Wikipedia: Raspberry Pi - Generations}
	\item Das Gerät sollte möglichst von vielen Herstellern vertriebenen Zigbee-Endgeräte verwalten können.
\end{itemize}
Die Beschaffung der Teile lief dank Online-Versandhandel relativ problemlos. Beim Zusammenbau haben wir den Prototyp lediglich mit dem Bildschirm, dem Pi und dem Zigbee-USB-Stick aufgebaut (vgl. \ref{hw_prototype}). Die beiden Standfüße stammen vom Hersteller des Bildschirms Sunfounder \footnote{Sunfounder: 10.1 Inch Touch Screen for Raspberry Pi(NEW) - 3D-printed Touch Screen Support} und wurden PETG auf dem Ender 3 von Manuel Starz gedruckt.Danach wurde die Softwareseite von Felix Kuschel bearbeitet, der die Software auf dem Raspberry OS verwendet hatte (vgl. \ref{hgw_software} \& \ref{software}).\par
Nachdem der Prototyp soweit funktionsfähig war, haben wir uns daran gemacht, das Präsentationsmodell zu entwerfen. Dabei ging es uns vornehmlich um die Unterbringung der geplanten Komponenten in einem simplen Gehäuse. Hierfür waren einige Verlängerungskabel nötig, um Stromzufuhr, Netzwerk und die Antenne des Zigbee-Sticks nach außen zu leiten. (vgl.\ref{hw_case_modellentwicklung} \& \ref{ku_produkt}). Das Gehäuse wurde dann nach dem Druck mit dem Bildschirm verklebt und die einzelnen Komponenten verbaut (vgl. \ref{hw_case_herstellung}). Nachdem das Gerät soweit fertig war, hat Felix Kuschel einen Versuchsaufbau zur Präsentation aufgebaut, der die Funktionsweise des Geräts in einem ,,typischen'' Heimnetz darstellen soll.