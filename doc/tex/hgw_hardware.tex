\subsection{Hardware}\label{hgw_hardware}
Bei der Entwicklung der Hardware haben wir für unser Projekt intern einige Rahmenbedinungen festgelegt:
\begin{itemize}
	\item das Projekt sollte nach Möglichkeit von durchschnittlich begabten Hobbyhandwerkern durchgeführt werden
	\item die Fertigstellung des Projekts sollte mit handelsüblichen Werkzeugen möglich sein
	\item die Grundplattform sollte der 2020 herausgebrachte Raspberry Pi 4 8GB sein\footnote{Wikipedia: Raspberry Pi - Generations}
	\item das Gerät sollte Zigbee-Endgeräte
verwalten können, da diese von vielen Herstellern vertriebenen werden.
\end{itemize}
\noindent Die Beschaffung der Teile lief dank Online-Versandhandel relativ problemlos. 
Beim Erstellen des Prototyps haben wir nur aus Bildschirm, dem Raspberry Pi und dem Zigbee-USB-Stick zusammengebaut (vgl. Abschnitt \ref{hw_prototype}: \nameref{hw_prototype}). 
Die Vorlage  Standfüße stammt vom Hersteller des Bildschirms Sunfounder \footnote{Sunfounder: 10.1 Inch Touch Screen for Raspberry Pi(NEW) - 3D-printed Touch Screen Support}.
Diese haben wir aus PETG\footnote{PETG: Glycol modifiziertes PET} auf dem 3D-Drucker Ender 3  selbst gedruckt. 
Im Folgenden haben wir uns mit der Software des Projekts befasst (vgl. Abschnitt \ref{hgw_software}: \nameref{hgw_software} und Kapitel \ref{software}: \nameref{software}).\par
\noindent Nachdem der Prototyp soweit funktionsfähig war, haben wir das Präsentationsmodell entworfen. 
Dabei ging es uns vornehmlich um die Unterbringung der geplanten Komponenten in einem einfach herzustellendem Gehäuse. 
Hierfür waren einige Verlängerung der Kabel nötig, um Stromzufuhr, Netzwerk und die Antenne des Zigbee-Sticks nach außen zu führen (vgl. Abschnitt \ref{hw_case_modellentwicklung}: \nameref{hw_case_modellentwicklung} und Abschnitt \ref{ku_produkt}: \nameref{ku_produkt}). \\
\noindent Das Gehäuse wurde nach dem Druck mit dem Bildschirm verklebt und die einzelnen Komponenten eingebaut (vgl. Abschnitt \ref{hw_case_herstellung}: \nameref{hw_case_herstellung}). 
Nach dem Zusammenbau haben wir einen Präsentationsaufbau, der die Funktionsweise unseres Prototyps in einem ,,typischen'' Heimnetzwerk darstellen soll (vgl. Abschnitt \ref{hw_testaufbau}: \nameref{hw_testaufbau}).\\
\noindent Näheres zur Umsetzung der Hardware im Kapitel \ref{hardware}: \nameref{hardware}.