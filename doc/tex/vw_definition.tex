\subsection{Definition Smart Home Zentrale}\label{vw_definition}
 	\begin{figure}[ht!]
 		\centering
 		\begin{subfigure}[t]{0.3\linewidth}
 			\centering
 			\includegraphics[width=0.9\textwidth]{img/alexa_echo_show8.jpg}
 			\caption[Amazon Alexa Echo Show 8]{Amazon Alexa Echo Show 8}
 			\label{fig:alexa-echo-show8}
 		\end{subfigure}
 		\hfill
 		\begin{subfigure}[t]{0.3\linewidth}
 			\centering
 			\includegraphics[width=0.9\textwidth]{img/google_nest_hub.png}
 			\caption[Google Nest Hub]{Google Nest Hub}
 			\label{fig:google-nest-hub}
 		\end{subfigure}
 		\hfill
 		\begin{subfigure}[t]{0.3\linewidth}
 			\centering
 			\includegraphics[width=0.9\textwidth]{img/glancr_smart_mirror.jpeg}
 			\caption[Glancr Smart Mirror]{Glancr Smart Mirror}
 			\label{fig:glancr-smart-mirror}
 		\end{subfigure}
 		\caption[Beispiele für Smart Home Zentralen]{Smart Home Zentralen}
 		\label{fig:smart-home-zentralen}
 	\end{figure}
\noindent Smart Home Zentralen, auch Smart Hubs oder Smart Mirrors genannt, sind Geräte, die als zentraler Knotenpunkt in einem Smart Home Netzwerk sitzen und dort Informationen verarbeiten, weiterleiten und darstellen können.\par
\noindent Diese Geräte werden von den meisten Herstellern mit und ohne Bildschirm geliefert, um entweder ein neues Smart Home aufzubauen oder ein bestehendes Smart Home zu erweitern. 
 	\begin{quote}
 		\color{quotetext}
 		Bei einem Smart-Home-Hub handelt es sich um eine schlaue Zentrale, durch die all deine intelligenten Geräte miteinander vernetzt werden – und dadurch erst wirklich ihren gesamten Leistungsumfang ausschöpfen.\footnote{Li (2017): Was ist ein Smart-Home-Hub? Alles über die intelligente Zentrale}
 	\end{quote}
\noindent Als Smart Home Zentrale können auch Software-Lösungen gezählt werden, die mit den im Netz befindlichen Smart Hubs kommunizieren und die Informationen mit Hilfe eines Web-Interfaces oder einer Smartphone-Anbindung darstellen und steuern können. 
Beispiele hierfür sind homeassistant.io, openHAB und Google Home.\par
% WARUM WILL DAS NIT ANGEZEIGT WERDEN!!!
\begin{figure}[h!tb]
 	\centering
 	\begin{subfigure}[b]{0.3\linewidth}
 		\centering
 		\includegraphics[width=0.7\textwidth]{img/home-assistant-io_logo.png}
 		\caption[Home Assistant Logo]{Home Assistant}
 		\label{fig:hassio-logo}
 	\end{subfigure}
 	\hfill
 	\begin{subfigure}[b]{0.3\linewidth}
 		\centering
 		\includegraphics[width=0.7\textwidth]{img/google-cast_logo.png}
 		\caption[Google Home Logo]{Google Home}
 		\label{fig:google-home-logo}
 	\end{subfigure}
 	\hfill
 	\begin{subfigure}[b]{0.3\linewidth}
 		\centering
 		\includegraphics[width=0.7\textwidth]{img/openhab_logo.png}
 		\caption[openHAB Logo]{openHAB}
 		\label{fig:openhab-logo}
 	\end{subfigure}
 	\caption[Beispiele für Smart Home Software]{Smart Home Software}
 	\label{fig:software-logos}
\end{figure}