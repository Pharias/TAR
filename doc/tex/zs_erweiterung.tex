\subsection{Erweiterungsmöglichkeiten}\label{zs_erweiterung}
Als eine Erweiterung des Projekts ist ein sogenannter Raspberry Pi HAT, eine aufsteckbare Erweiterung für den Raspberry Pi geplant.
% Nach Fortschritt des Projekts kam bei einer Rücksprache mit unserem Projektbetreuer die Idee auf, einen Raspberry Pi HAT speziell für das Projekt zu entwickeln. 
Die Idee dazu entstand in Rücksprache mit unserem Projektbetreuer, der den HAT darüber hinaus als Unterrichtsmaterial im Laborunterricht einsetzten wollte.
Dieser sollte die Hardware des Projekts, falls möglich auf einer Platine vereinen.\par
\noindent Der HAT sollte folgende Bauteile enthalten:
\begin{itemize}
	\item ZigBee-Controller und Antenne
	\item NFC-Controller und Antenne
	\item RGB-LED zur Statusanzeige
	\item Anschluss für Lüfter
	\item Sensoren für:
	\begin{itemize}
		 \item Luftfeuchtigkeit
		 \item Temperatur
		 \item Luftdruck
	\end{itemize}
	\item Pins zum Anschluss an Versuchsaufbau für Laborgebrauch in der Schule
\end{itemize}
Der HAT wurde aber aufgrund der aktuellen Pandemie-Situation und den damit verbundenen Liefer- und Zollschwierigkeiten verworfen.
Die Planung des HAT ist daher nicht vollständig, kann aber bei Bedarf ergänzt werden (vgl. Abschnitt \ref{hw_hat}: \nameref{hw_hat}).