\subsection{Raspberry Pi 4}
Als Basis für die Smart Home Zentrale haben wir einen Raspberry Pi in der Version 4 mit 8 GB gewählt, um als von anderen Servern unabhängige Plattform zu agieren. Der Raspberry Pi 4 ist mit einem ARM Cortex-A72 Prozessor ausgestattet, der über 4 Kerne verfügt. Darüber hinaus verfügt der Raspberry Pi 4 über einen Gigabit-Netzanschluss. Darüber hinaus ist der Raspberry Pi in der Bastlerszene weit verbreitet und dient bei anspruchsvolleren Projekten als Kern\par
\begin{quote}
 		\color{quotetext}
 		Der Raspberry Pi hat sich seit der ersten Veröffentlichung Anfang 2012 weltweit schon millionenfach verkauft und erfreut sich immer noch größter Beliebtheit. Denn viele Raspberry Pi User haben nicht nur einen Einplatinen-Computer zu Hause, sondern teils 4-5 Stück.
Der eine fungiert als HD-Mediaplayer mit externer Festplatte für das heimische Kino oder als Internetradio mit Display, der nächste als Webcam-Server für die Kameraüberwachung mit Livestream auf das Handy, dann noch einer für die Hausautomatisierung wie bspw. die Heizungs- oder Lichtsteuerung und noch einer als einfacher WLAN-Druckerserver oder als Mini-Computer zum allgemeinen Surfen im Internet, um ein paar wenige Anwendungsszenarien zu nennen.
Sie merken, der kleine ,,Tausendsassa'' kann nicht nur viel, sondern ist zudem auch noch extrem günstig und eben das macht den Reiz aus. Lediglich den Hang zum Programmieren sollten Sie mitbringen und selbst nicht mal das, denn Sie können auch einfach nach Anleitung aus Foren oder Büchern nachprogrammieren, oder ganz bequem ein fertiges Image auf den Raspberry Pi installieren - trauen Sie sich! 
\footnote{reichelt.de: Produktbeschreibung des Raspberry Pi 4}
\end{quote}