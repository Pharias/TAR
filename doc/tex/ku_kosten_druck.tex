\subsubsection{Kostenberechung für die 3D-Druckteile}\label{ku_kosten_druck}
\noindent Die Kosten lassen mit folgender Formel darstellen:\\
\[
K_{gesamt} =  (K_{Material/g} \cdot M_{Material}) +  (T_{Druck} \cdot  K_{kWh} \cdot V_{\emptyset/h})
\]
Die Kosten der 3D-Druckteile lassen sich in zwei Einzelpositionen aufteilen: Material und Energie.
Die Materialkosten lassen sich leicht erfassen, da wir die Kosten der Spule Filament mit 24,95 \euro{} und einem Gewicht von 800 g Filament auf der Spule gegeben haben.
So ergibt sich ein Preis von 0,03(12...)\euro{} ($K_{Material/g}$) pro Gramm verwendetem Filament.
Der Druck der Gehäuseteile verbrauchte 423 g Material ($M_{Material}$), das heißt, dass die Materialkosten sich auf 13,19 \euro{} belaufen.\par
\noindent Die Energiekosten zu berechnen ist etwas komplexer.
Der 3D-Drucker verbraucht circa 120 W pro Stunde($V_{\emptyset/h}$)\footnote{Robert (2019): Antwort auf Creality Ender 3 printer power consumption? - 3dprinting Stack Exchange}, was bei einem Strompreis von 0,31 \euro{} / kWh ($K_{kWh}$) bedeuten würde, dass wir pro Stunde etwa 0,0372 \euro{} Stromkosten haben. 
Bei einer Gesamtdruckzeit von 36 Stunden und 36 Minuten($T_{Druck}$) ergeben sich also Stromkosten von 1,37 \euro{}.\par
\noindent Addiert man nun die Kosten für Material und Energie, so ergibt sich ein Fertigungspreis von 12,98 \euro{} ($K_{gesamt}$). 
Dabei ist der Zeitaufwand für die Veredelung, also Schleifen und Lackieren, noch nicht mit einkalkuliert.\par