\subsubsection{Modellentwicklung am Objekt}
Nachdem das Testmodell (vgl. Abbildung \ref{fig:print-case-test_04}) nicht zu 100 \% gepasst hat, hat Manuel Starz die Abmessungen neu geklärt und diese in Fusion 360 übertragen (vgl. \ref{case_footprint}). Um Material für den 3D-Druck zu sparen, wurde die Zeichnung dann im Maßstab 1:1 auf Papier gedruckt, ausgeschnitten und angelegt.\par
Da hier einige Maße noch nicht gestimmt haben, hat Manuel Starz den Plan überarbeitet (vgl. \ref{case_footprint_final}). Diese neuen Bemaßungen waren dann korrekt.\par
Daraufhin wurde dann die Zeichnung in zwei eigenständige Dateien gesplittet, um die linke und die rechte Seite des Gehäuses zu konstruieren.\par
